\documentclass[a4paper]{article}

\usepackage[utf8]{inputenc}
\usepackage{graphicx}
\usepackage[frenchb]{babel}
\usepackage{amsmath}

\begin{document}

\title{VISA -- TP2: mise en correspondance stéréoscopique}
\author{Arnaud Cojez}
\date{mardi 4 octobre 2016}

\maketitle

\newpage
\tableofcontents
\newpage
%----------------------------------------------------------------------------------------
%	INTRODUCTION
%----------------------------------------------------------------------------------------

\section{Introduction}

\subsection{Motivation}

La photographie permet de récupérer et de conserver les données liées à une scène, à un instant donné. Cependant, les données récupérées correspondent à la projection de la scène sur un plan 2D. Par conséquent, nous observons une perte d'information, notamment au niveau de la profondeur.\\

Il existe différents moyens de pallier à ce problème. Ceux-ci sont plus ou moins efficaces, et plus ou moins simples à mettre en œuvre.\\
On peut diviser ces méthodes en 2 catégories :
\begin{itemize}
    \item Éparses : Une sélection de points est utilisée pour la mise en correspondance entre 2 images;
    \item Denses : Tous les points d'une image sont mis en relation avec ceux d'une seconde image.
\end{itemize}

Parmi les méthodes éparses, nous trouvons la méthode de la stéréoscopie. C'est celle qui nous intéresse ici.

\subsection{La mise en correspondance stéréoscopique}

Cette méthode consiste à interpréter 2 (ou plus) projections d'une même scène, avec des points de vue différents afin d'estimer la profondeur des objets dans la-dite scène.\\

L'avantage de cette méthode est qu'elle est peu complexe (calculs sur quelques points remarquables sélectionnés) et facile à mettre en œuvre (2 images sont suffisantes afin d'avoir des résultats satisfaisants).\\

Cette méthode se découpe en 3 étapes principales :
\begin{enumerate}
  \item Identification des points et formes remarquables : appelés \em{features} ;
  \item recherche de correspondance et appairage entre les features trouvées plus tôt ;
  \item estimation de la profondeur et reconstitution 3D par triangulation.
\end{enumerate}

Ce sont ces étapes que nous allons décrire dans ce document.

\clearpage
%----------------------------------------------------------------------------------------
%	CALCUL DE LA MATRICE FONDAMENTALE
%----------------------------------------------------------------------------------------

\section{Matrice fondamentale}

\subsection{Matrice et produit vectoriel}

Pour calculer le point d'intersection de deux droites, on doit se servir d'une opération appelée \em{produit vectoriel}.\\
Celui-ci peut-être mis sous la forme d'une matrice de transformation :
\begin{equation}
  p \times q = p^\times \cdot q =>
\end{equation}
\begin{equation}
  p^\times =
  \begin{bmatrix}
    0 & -p_z & p_y\\
    p_z & 0 & -p_x\\
    -p_y & p_x & 0
  \end{bmatrix}
\end{equation}

Nous aurons besoin de cette matrice de transformation linéaire (ou homographie) afin de calculer la matrice fondamentale.

\subsection{Calcul de la matrice fondamentale}

\subsection{Équations des droites épipolaires}

\clearpage
%----------------------------------------------------------------------------------------
%	EXTRACTION DES COINS
%----------------------------------------------------------------------------------------

\section{Extraction des coins}


\clearpage
%----------------------------------------------------------------------------------------
%	CALCUL DES DISTANCES
%----------------------------------------------------------------------------------------

\section{Calcul des distances}


\clearpage
%----------------------------------------------------------------------------------------
%	MISE EN CORRESPONDANCE
%----------------------------------------------------------------------------------------

\section{Mise en correspondance}


\clearpage
%----------------------------------------------------------------------------------------
%	RÉSULTATS OBTENUS
%----------------------------------------------------------------------------------------

\section{Résultats}

\clearpage
%----------------------------------------------------------------------------------------

%----------------------------------------------------------------------------------------
%	CONCLUSION
%----------------------------------------------------------------------------------------

\section{Conclusion}

\clearpage
%----------------------------------------------------------------------------------------

\end{document}
