\documentclass[a4paper]{article}

\usepackage[utf8]{inputenc}
\usepackage{graphicx}
\usepackage[frenchb]{babel}
\usepackage{amsmath}

\begin{document}

\title{VISA -- TP2: mise en correspondance stéréoscopique}
\author{Arnaud Cojez}
\date{mardi 4 octobre 2016}

\maketitle

\newpage
\tableofcontents
\newpage
%----------------------------------------------------------------------------------------
%	INTRODUCTION
%----------------------------------------------------------------------------------------

\section{Introduction}

\subsection{Motivation}
La photographie permet de récupérer et de conserver les données liées à une scène, à un instant donné. Cependant, les données récupérées correspondent à la projection de la scène sur un plan 2D. Par conséquent, nous observons une perte d'information, notamment au niveau de la profondeur.\\

Il existe différents moyens de pallier à ce problème. Ceux-ci sont plus ou moins efficaces, et plus ou moins simples à mettre en œuvre.


\subsection{La Méthode de Zhang}

\clearpage
%----------------------------------------------------------------------------------------
%	DÉTERMINATION DES CONTRAINTES À PARTIR DE L'HOMOGRAPHIE
%----------------------------------------------------------------------------------------

\section{Homographie et contraintes}

\clearpage
%----------------------------------------------------------------------------------------
%	DÉTERMINATION DE LA MATRICE INTRINSÈQUE
%----------------------------------------------------------------------------------------

\section{Matrice Intrinsèque}


\clearpage
%----------------------------------------------------------------------------------------
%	DÉTERMINATION DE LA MATRICE EXTRINSÈQUE
%----------------------------------------------------------------------------------------

\section{Matrice Extrinsèque}


\clearpage
%----------------------------------------------------------------------------------------
%	RÉSULTATS OBTENUS
%----------------------------------------------------------------------------------------

\section{Résultats}

\subsection{Matrice Intrinsèque}

\subsection{Matrices Extrinsèques}


\clearpage
%----------------------------------------------------------------------------------------
%	CALIBRATION FOCALE
%----------------------------------------------------------------------------------------

\section{Zhang simplifié}




\clearpage
%----------------------------------------------------------------------------------------

%----------------------------------------------------------------------------------------
%	CONCLUSION
%----------------------------------------------------------------------------------------

\section{Conclusion}

\clearpage
%----------------------------------------------------------------------------------------

\end{document}
